\section{Persönliches}

Der Laborversuch ist interessant. Ein anspruchsvolles Nebengefecht ist die Wahrung des Überblicks über Messungen und physikalische Verhältnisse. Die Verbindung von behandelter Theorie und Praxis wird nicht erleichtert, solange Unsicherheiten vorhanden sind.\\
\\
Die Laborzeit für die Durchführung aller in der Versuchsanleitung gestellten Aufgaben ist eher knapp bemessen. Die Zeit hat uns nicht gereicht, alle Aufgaben zu bearbeiten.\\
\\
Die Parametereinstellungen am Wechselrichter während des Versuches haben relativ viel Zeit in Anspruch genommen. Parameter sind nur in linearen Schritten mit kleiner Auflösung verstellbar. Dies erfordert langes drücken und gedrückt halten des Tasters. Als Optimierung könnte die Schrittweite erhöht werden, sobald der Benutzer den Taster eine längere Zeit drückt. Durch diese Optimierung sind Parameteränderungen schneller durchführbar und damit die Zeit sinnvoller nutzbar.\\
\\
Wünschenswert ist ein Elektroschema bezüglich Verkabelung des Versuchsaufbaus. Dies kann den Einstieg in den Labornachmittag erleichtern, wodurch man Zeit gewinnt.

\newpage
