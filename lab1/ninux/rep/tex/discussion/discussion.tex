\section{Diskussion}

%\todo{Die wichtigsten Ergebnisse der Versuche werden in Worte qualitativ
%	beschrieben und bewertet. Unstimmigkeiten zwischen den Ergebnissen
%	einerseits und den allfällig berechneten / simulierten Grössen
%	andererseits sollen diskutiert werden, und mögliche Ursachen als
%	Hypothesen formuliert werden.}

Gemäss der festgehaltenen Resultate zeigen die Simulationen und Messungen
geringfügige Abweichungen. Diese sollen im Folgenden kurz diskutiert
werden.

\subsection{Streuparameter $S_{11}$ (Reflexion)}
Die Simulation zeigt eine geringfügige Abweichung von der gweünschten
Frequenz $f$, obwohl die berechneten Masse gemäss den Vorgaben verwendet
worden sind. Das Ergebnis der Simulation deutete dabei auf geringfügig zu
lange Antenn hin. Die Messung des Streuparameters $S_{11}$ mit dem
Vector-Network Analyzer zeigte ebenfalls eine geringfügige Abweichung des
Resonanzpunktes auf. Da die Stablänge sukzessive angepasst wurde gemäss den
Ergebnissen der Messung mit dem Vector-Network Analyzer, ist diese Abweichung
in keinem formalen Fehler zu suchen, sondern lediglich als das Ergebnis der
mechanischen \emph{Trimmgenauigkeit} zu identifizieren. Als formalen
und systematischen Fehler kann der Einfluss der Annahme der
Freiraumausbreitungsgeschwindikkeit $c$ erachtet werden (siehe
Abschnitt \ref{sec:calculation_frequency}), deren Einfluss in der
vorliegenden Arbeit nicht weiter untersucht wird.

\subsection{Impedanz}
Die Impedanz zeigt ebenfalls nur geringfügige Abweichungen zwischen
der Simulation und der Messung. Die genaue Ursache konnte nicht
ermittelt werden. An dieser Stelle ist festzuhalten, dass sich
der Aufbau des Simulationsmodells und des realisierten Modells deutlich
unterscheidet. Zum einen verfügt das Simulationsmodell mit der
eingesetzten Quelle über eine ideale Anbindung zwischen der Groundplane
und dem Stab, was im realen Modell durch Lötstellen und einen SMA
Stecker realisiert ist. Zum anderen wurde der verwendete Stecker
nicht für die Kalibrierung des Messsystems berücksichtigt, wodurch
eine geringere Messungenauigkeit zu erwarten ist. Weiter ist zu
berücksichtigen, dass das Simulationmodell ohne Verluste simuliert
wurde, wodurch anzunehmen ist, dass das Verhalten der simulierten
Antenne sich weiter vom Verhalten des realen Modells entfernt.

\subsection{Abstrahlung}
Die Abstrahlcharakteristik aus Simulation und Messung zeigt wesentliche
Unterschiede auf betreffend der Symmetrie und dem Winkel der maximalen
Abstrahlung. Hier ist davon auszugehen, dass die Eigenschaften der
realisierten Groundplane einen massgebenden Einfluss auf die beobachteten
Diskrepanzen hat. Insbesondere die gemessene \emph{unruhe} der Abstrahlung
in Richtung der Goundplane zeigen, dass in Richtung der Groundplane ein
signifikanter Einfluss auf die Felder der Antenne vorliegt. Zur
Verbesserung und Verifikation dieser Vermutung ist die Messung mit einer
grösseren Groundplane zu wiederholen, welche in der vorliegenden Arbeit
nicht behandelt wird. 
