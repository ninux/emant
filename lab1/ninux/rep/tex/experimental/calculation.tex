\subsection{Berechnungen}

\subsubsection{Wahl der Betriebsfrequenz}
\label{sec:calculation_frequency}

Um eine möglichst kompakte Antenne zu erhalten, ist eine hohe
Betriebsfrequenz zu wählen, denn die geometrischen Abmessungen
der Monopolantenne sind umgekehrt proportional zur Frequenz
respektive proportional zur Wellenlänge. In der vorliegenden
Arbeit wird auf eine Korrektur der Geschwindigkeit $c$ verzichtet
(Freiraumausbreitung).

\begin{equation}
	l \sim \lambda = \frac{c}{f}
\end{equation}

Um also eine Monopolantenne zu erhalten, deren Stablänge $l_s$
und Kantenlänge $l_k$ möglichst klein sind, ist eine entsprechend
hohe Frequenz $f$ zu wählen. Die Betriebsfrequenz wird gewählt mit
80\% des erlaubten Spektrums.

\begin{equation}
	f
	= 0.8 \cdot f_{max}
	= 0.8 \cdot \SI{5}{\giga\hertz}
	= \SI{4}{\giga\hertz} 
\end{equation}

\subsubsection{Monopolantenne auf quadratischer Groundplane}

\begin{figure}[h!]
	\centering
	\def\svgwidth{0.5\textwidth}
	\input{../fig/gph/monopol_a.pdf_tex}
	\caption{Modellzeichnung der Monopolantenne mit quadratischer
		Groundplane.}
\end{figure}

Für die Dimensionierung der Antenne wird die Wellenlänge benötigt,
welche sich mit der gewählten Frequenz $f$ und der Lichtgeschwindigkeit
$c$ ermitteln lässt.

\begin{equation}
	\lambda
	= \frac{c}{f}
	\approx \frac{\SI{3d8}{\meter\per\second}}{\SI{4}{\giga\hertz}}
	= \SI{75d-3}{\meter} 
\end{equation}

Aus der Aufgabenstellung geht hervor, dass die Groundplane mit einer
Diagonale $l_d$ von $\frac{\lambda}{\sqrt{2}}$ zu erstellen ist
\cite[S.1, 2a]{lab1}. Für eine quadratische Groundplane ergibt sich
die Kantenlänge $l_k$ der Groundplane somit zu $\frac{\lambda}{2}$.

\begin{equation}
	l_d
	= \sqrt{{l_k}^2 + {l_k}^2}
	= \sqrt{2 \cdot ({l_k}^2)}
	= \sqrt{2} \cdot l_k
\end{equation}

\begin{equation}
	l_k
	= \frac{l_d}{\sqrt{2}}
	= \frac{\left(\frac{\lambda}{\sqrt{2}}\right)}{\sqrt{2}}
	= \frac{\lambda}{\sqrt{2} \cdot \sqrt{2}}
	= \frac{\lambda}{2}
	= \frac{\SI{75d-3}{\meter}}{2}
	= \SI{37.5d-3}{\meter}
\end{equation}

Die Stablänge $l_s$ der Antenne ist analog zur Groundplane mit der
Frequenz $f$ respektive der Wellenlänge $\lambda$ zu ermitteln.

\begin{equation}
	l_s
	= \frac{\lambda}{4}
	= \frac{\SI{75d-3}{\meter}}{4}
	= \SI{18.8d-3}{\meter}
\end{equation}


