\section{Einführung}

%\todo{Abschnitt umfasst die angewandten Methoden und Prozeduren sowie
%	wichtige Zusammenhänge. Maximal zwei Seiten.}

Für die Messung und Realisierung einer Monopol-Antenne wurde ein
Simulationsmodell erstellt mit der Software
Empire XPU. Dieses wurde gemäss vorgegebenen Parametern formal
entworfen und simuliert. Ausgewertet wurden dabei der
Streu- respektive Reflexionsparameter $S_{11}$, die Impedanz
sowie die Abstrahlcharakteristik der Antenne.

Gemäss dem simulierten Modell wurde eine Antenne realisiert.
An diesem wurden mittels eines Vector-Network Analyzer der
$S_{11}$ Parameter sowie die Impedanz gemessen. Darauffolgend
wurde die Abstrahlcharakteristik gemessen mittels eines
Nahfeldmessgerätes (StarLab), welches aus dem gemessen
Nahfeld das Fernfeldverhalten der Antenne bestimmt.

Das Vorgehen und die erzielten Ergebnisse werden in der
vorliegenden Arbeit vorgestellt, zusammengefasst und
diskutiert.
