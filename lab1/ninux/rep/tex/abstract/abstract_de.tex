% https://www.ncbi.nlm.nih.gov/pmc/articles/PMC3136027/
\thispagestyle{empty}
\begin{abstract}
Eine Monopolantenne besteht aus einem Stab, einer Groundplane und einer
Einspeisung. In der vorliegenden Arbeit soll eine Monopolantenne
als Simulationmodell und realer Aufbau realisiert, gemessen und
bewertet werden.

Für die Simulation wurde ein Modell mit der Software Empire XPU erstellt.
Mit dem erstellten Simulationmodell wurde der Streu- respektive
Reflexionsparameter, die Impedanz sowie die Abstrahlcharakteristik
der Antenne ermittelt.

In einem weiteren Schritt wurde gemäss dem Simulationsmodell ein reales
Modell der Antenne realisiert. Von diesem Modell wurden mit einem
Vector-Network Analyzer der Streu- und Reflexionsparameter $S_{11}$
sowie die Impedanz gemessen.

In einem letzten Schritt wurde die Abstrahlcharakteristik des
realisierten Modells gemessen. Dies erfolgte mittels eines
Nahfeldmessgerätrs und einer Transformation des gemessenen Nahfelds
ins Fernfeld der Antenne.

Die durchgeführten Messungen zeigen, dass das Verhalten der realisierten
Antenne geringfügig von den Ergebnissen der Simulation abweichen. Die
erzeilten Ergebnisse deuten darauf hin, dass die Diskrepanzen zwischen
der Simulation und dem realen Aufbau durch die verlustfreie Simulation
und die geringe geometrische Dimension der realisierten Groundplane 
verursacht werden.
\end{abstract}
