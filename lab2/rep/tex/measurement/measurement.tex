\section{Messungen}
Die entwickelte Platine

\begin{figure}[h!]
	\centering
	\includegraphics[width=0.5\textwidth, angle = 180]{../fig/plt/Platine.JPG}
	\caption{Die entwickelte Platine}
	\label{fig:Platine}
\end{figure}


\subsection{Messungen mit dem Netzwerkanalyzer}

Marke: Rhode$\&$Schwartz\\
Inventarnr.: 14006

\vspace*{3mm}

Vor dem Einsatz des Netzwerkanalyzers wurde dieser kalibiert.

\begin{figure}[h!]
	\centering
	\includegraphics[width=0.8\textwidth]{../fig/plt/Calib.JPG}
	\caption{Kalibrierung des Netzwerkanalyzers}
	\label{fig:calib}
\end{figure}

Mit dem Netzwerkanalyzer wurden Messungen ohne und mit Gehäuse gemacht
bei \SI{2.4}{\giga\hertz} und \SI{5}{\giga\hertz}. In diesem Bericht
wird der Fokus auf die Messung mit Gehäuse gelegt. Zum einen Überblick
über das gesamte Frequenzspektrum zu vermitteln. Abbildungen x zeigen
einen Überblick über das gesamte Frequenzspektrum von
\SI{1.75}{\giga\hertz} bis \SI{5.75}{\giga\hertz}. Im Abschnitt x und
Abschnitt y wird auf die Messungen bei den spezifischen Designfrequenzen
eingegangen.

\begin{figure}[h!]
	\begin{center}
		\begin{subfigure}[t]{0.49\textwidth}
			\begin{center}
				\includegraphics[width=1\textwidth]{../fig/plt/S11_WITH_FULL.PNG}
				\caption{Reflexionskoeffizient}
				\label{fig:S11_with_full}
			\end{center}
		\end{subfigure}
		\begin{subfigure}[t]{0.49\textwidth}
			\begin{center}
				\includegraphics[width=1\textwidth]{../fig/plt/IMP_WITH_FULL.PNG}
				\caption{Impedanz}
				\label{fig:Imp_with_full}
			\end{center}
		\end{subfigure}
		\caption{Überblick über gesamtes Frequenzspektrum}
		\label{fig:full}
	\end{center}
\end{figure}


\clearpage
\subsubsection{Reflexionskoeffizient S11}

Bei 2.4 GHz zeigte sich ein äusserst ausgeprägter Reflexionskoeffizent kleiner als -50dB

\begin{figure}[h!]
	\begin{center}
		\begin{subfigure}[t]{0.49\textwidth}
			\begin{center}
				\includegraphics[width=1\textwidth]{../fig/plt/S11_WITH_2_4.PNG}
				\caption{bei 2.4 GHz}
				\label{fig:S11_with_full_2.4}
			\end{center}
		\end{subfigure}
		\begin{subfigure}[t]{0.49\textwidth}
			\begin{center}
				\includegraphics[width=1\textwidth]{../fig/plt/S11_WITH_5_0.PNG}
				\caption{bei 5.0 GHz}
				\label{fig:S11_with_full_5.0}
			\end{center}
		\end{subfigure}
		\caption{Reflexionskoeffizient S11}
		\label{fig:S11_each}
	\end{center}
\end{figure}

\subsubsection{Impedanz}

\begin{figure}[h!]
	\begin{center}
		\begin{subfigure}[t]{0.49\textwidth}
			\begin{center}
				\includegraphics[width=1\textwidth]{../fig/plt/IMP_WITH_2_4.PNG}
				\caption{bei 2.4 GHz, !Achtung falsche Skala!}
				\label{fig:Imp_with_full_2.4}
			\end{center}
		\end{subfigure}
		\begin{subfigure}[t]{0.49\textwidth}
			\begin{center}
				\includegraphics[width=1\textwidth]{../fig/plt/IMP_WITH_5_0.PNG}
				\caption{bei 5.0 GHz}
				\label{fig:Imp_with_full_5.0}
			\end{center}
		\end{subfigure}
		\caption{Impedanzen}
		\label{fig:Imp_each}
	\end{center}
\end{figure}

\newpage
\subsection{Messungen mit dem Starlab}
\begin{figure}[h!]
	\centering
	\def\svgscale{0.75}
	\input{../fig/gph/starlab_setup_annotated.pdf_tex}
	\caption{Aufbau der Messung im StarLab.}
	\label{fig:LaptopimStarlab}
\end{figure}

%\begin{figure}[h!]
%	\centering
%	\includegraphics[width=0.6\textwidth]{../fig/plt/LaptopimStarLab.JPG}
%	\caption{DUT eingesteckt am Laptop im LinearScanner}
%	\label{fig:LaptopimStarlab}
%\end{figure}

\clearpage
\subsubsection{Abstrahlung 3D}

\begin{figure}[h!]
	\centering
	\begin{subfigure}[t]{0.35\textwidth}
		\begin{center}
			\includegraphics[width=1\textwidth]{../fig/plt/star_lab_2ghz4_xy_reduced.png}
			\caption{$\vec{E_{tot}}$ XY \SI{2.4}{\giga\hertz}}
		\end{center}
	\end{subfigure}
	\begin{subfigure}[t]{0.35\textwidth}
		\begin{center}
			\includegraphics[width=1\textwidth]{../fig/plt/star_lab_5ghz0_xy_reduced.png}
			\caption{$\vec{E_{tot}}$XY \SI{5.0}{\giga\hertz}}
		\end{center}
	\end{subfigure}
	
	\begin{subfigure}[t]{0.35\textwidth}
		\begin{center}
			\includegraphics[width=1\textwidth]{../fig/plt/star_lab_2ghz4_xz_reduced.png}
			\caption{$\vec{E_{tot}}$XZ \SI{2.4}{\giga\hertz}}
		\end{center}
	\end{subfigure}
	\begin{subfigure}[t]{0.35\textwidth}
		\begin{center}
			\includegraphics[width=1\textwidth]{../fig/plt/star_lab_5ghz0_xz_reduced.png}
			\caption{$\vec{E_{tot}}$XZ \SI{5.0}{\giga\hertz}}
		\end{center}
	\end{subfigure}

	\begin{subfigure}[t]{0.35\textwidth}
		\begin{center}
			\includegraphics[width=1\textwidth]{../fig/plt/star_lab_2ghz4_yz_reduced.png}
			\caption{$\vec{E_{tot}}$YZ \SI{2.4}{\giga\hertz}}
		\end{center}
	\end{subfigure}
	\begin{subfigure}[t]{0.35\textwidth}
		\begin{center}
			\includegraphics[width=1\textwidth]{../fig/plt/star_lab_5ghz0_yz_reduced.png}
			\caption{$\vec{E_{tot}}$YZ \SI{5.0}{\giga\hertz}}
		\end{center}
	\end{subfigure}
	\caption[3D Abstrahlung gemessen mit StarLab]{
		3D Abstrahlung gemessen mit StarLab. Die Abbildung
		zeigt die Abstrahlung als eingefärbte Kugeln in den
		drei Grundansichten XY, XZ und YZ für die beiden
		Frequenzen \SI{2.4}{\giga\hertz} und
		\SI{5.0}{\giga\hertz}.}
\end{figure}

%\clearpage
%\begin{figure}[h!]
%	\begin{center}
%		\begin{subfigure}[t]{0.49\textwidth}
%			\begin{center}
%				\includegraphics[width=1\textwidth]{../fig/plt/2_4GHz_justsphere.jpg}
%				\caption{Kugel 2.4 GHz}
%				\label{fig:sphere_2ghz4}
%			\end{center}
%		\end{subfigure}
%		\begin{subfigure}[t]{0.49\textwidth}
%			\begin{center}
%				\includegraphics[width=1\textwidth]{../fig/plt/5GHz_just_sphere.jpg}
%				\caption{Kugel 5.0 GHz}
%				\label{fig:sphere_5ghz0}
%			\end{center}
%		\end{subfigure}
%		\caption{Starlab Messung Etot}
%		\label{fig:starlab_etot_sphere}
%	\end{center}
%\end{figure}

\clearpage
\subsubsection{Abstrahlung 1D}


\begin{figure}[h!]
	\centering
	\begin{subfigure}[b]{0.48\textwidth}
		\includegraphics[width=1\textwidth]{../fig/plt/2G4_90phi_etot_dB.JPG}
		\caption{$\vec{E_{tot}}$ mit $\varphi=\SI{90}{\degree}$ bei \SI{2.4}{\giga\hertz}}
	\end{subfigure}
	\begin{subfigure}[b]{0.48\textwidth}
		\includegraphics[width=1\textwidth]{../fig/plt/5G0_90phi_etot_dB.JPG}
		\caption{$\vec{E_{tot}}$ mit $\varphi=\SI{90}{\degree}$ bei \SI{5.0}{\giga\hertz}}
	\end{subfigure}

	\begin{subfigure}[b]{0.48\textwidth}
		\includegraphics[width=1\textwidth]{../fig/plt/2G4_90phi_ephi_amp_dB.JPG}
		\caption{$\vec{E_{\varphi}}$ mit $\varphi=\SI{90}{\degree}$ bei \SI{2.4}{\giga\hertz}}
	\end{subfigure}
	\begin{subfigure}[b]{0.48\textwidth}
		\includegraphics[width=1\textwidth]{../fig/plt/5G0_90phi_ephi_amp_dB.JPG}
		\caption{$\vec{E_{\varphi}}$ mit $\varphi=\SI{90}{\degree}$ bei \SI{5.0}{\giga\hertz}}
	\end{subfigure}

	\begin{subfigure}[b]{0.48\textwidth}
		\includegraphics[width=1\textwidth]{../fig/plt/2G4_90phi_etheta_amp_dB.JPG}
		\caption{$\vec{E_{\theta}}$ mit $\varphi=\SI{90}{\degree}$ bei \SI{2.4}{\giga\hertz}}
	\end{subfigure}
	\begin{subfigure}[b]{0.48\textwidth}
		\includegraphics[width=1\textwidth]{../fig/plt/5G0_90phi_etheta_amp_dB.JPG}
		\caption{$\vec{E_{\theta}}$ mit $\varphi=\SI{90}{\degree}$ bei \SI{5.0}{\giga\hertz}}
	\end{subfigure}
	\caption[1D Messungen mit StarLab für $\vec{E_{tot}}$, $\vec{E_{\varphi}}$ und $\vec{E_{\theta}}$]{
		1D Messungen mit StarLab für $\vec{E_{tot}}$, $\vec{E_{\varphi}}$ und $\vec{E_{\theta}}$.
		Die Abbildung zeigt die gemessenen Verläufe für die
		verschiedenen Vektoren bei fixiertem
		$\varphi = \SI{90}{\degree}$.}
\end{figure}



%\begin{figure}[h!]
%	\begin{center}
%		\begin{subfigure}[t]{0.49\textwidth}
%			\includegraphics[width=1\textwidth]{../fig/plt/2G4_90phi_etot_dB.JPG}
%			\caption{\SI{2.4}{\giga\hertz}}
%		\end{subfigure}
%		\begin{subfigure}[t]{0.49\textwidth}
%			\includegraphics[width=1\textwidth]{../fig/plt/5G0_90phi_etot_dB.JPG}
%			\caption{\SI{5.0}{\giga\hertz}}
%		\end{subfigure}
%		\caption[Auswertung der Abstrahlungscharakteristik]{
%			Auswertung der Abstrahlungscharakteristik.
%			Die Abbildung zeigt die gemessenen Werte von $\vec{E}_{tot}$
%			bei fixiertem $\varphi = \SI{90}{\degree}$.}
%		\label{fig:starlab_etot_curve}
%	\end{center}
%\end{figure}
%
%\begin{figure}[h!]
%	\centering
%	\includegraphics[width=0.8\textwidth]{../fig/plt/comparison_l4_pcb_v2c_laptop_1a_105_etot_phi90_2ghz4.png}
%	\caption{$\vec{E}_{\varphi}$ \;\; \SI{90}{\degree} \\
%		\hspace*{107pt}oben: simuliert\\
%		\hspace*{115pt}unten: gemessen}
%	\label{fig:E_tot_90}
%\end{figure}

\clearpage
\subsubsection{Optimale Abstrahlung}

\begin{figure}[h!]
	\begin{subfigure}[t]{0.49\textwidth}
		\includegraphics[width=1\textwidth]{../fig/plt/2_4GHz_justsphere.jpg}
		\caption{Kugel 2.4 GHz}
	\end{subfigure}
	\begin{subfigure}[t]{0.49\textwidth}
		\includegraphics[width=1\textwidth]{../fig/plt/5GHz_just_sphere.jpg}
		\caption{Kugel 5.0 GHz}
	\end{subfigure}
	
	\begin{subfigure}[t]{0.49\textwidth}
		\includegraphics[width=1\textwidth]{../fig/plt/2_4GHz_E_tot_curve.jpg}
		\caption{Kurve 2.4 GHz}
	\end{subfigure}
	\begin{subfigure}[t]{0.49\textwidth}
		\includegraphics[width=1\textwidth]{../fig/plt/5GHz_E_tot_curve.jpg}
		\caption{Kurve 5.0 GHz}
	\end{subfigure}
	\caption[Optimale Abstrahlung ausgewertet mit dem Antenna Analyser]{
		Optimale Abstrahlung ausgewertet mit dem Antenna Analyser.
		Die Abbildung zeigt den jeweiligen Punkt mit optimaler
		Abstrahlung für \SI{2.4}{\giga\hertz} und \SI{5.0}{\giga\hertz}
		als schwarzen Punkt in den 3D Darstellungen (a,b). Der durch den
		jeweiligen Punkt verlaufende Ring (gestrichtelt) zeigt die
		ausgewertete Abstrahlung, welche als eindimensionaler Verlauf
		von $\vec{E_{tot}}$ dargestellt ist (c,d).}
	\label{fig:etot_max_measured}
\end{figure}

Die optimale Abstrahlung wurde mit dem Antenna Analyser ermittelt.
Die Abbildung \ref{fig:etot_max_measured} zeigt die dabei festgestellten
Ergebnisse.

\clearpage
\section{Vergleich Simulation und Messung}

\subsection{Abstrahlung 3D}

\begin{figure}[h!]
	\centering
	\begin{subfigure}[b]{0.48\textwidth}
		\includegraphics[width=1\textwidth]{../fig/plt/crazy_stuff_l4_pcb_v2c_laptop_1a_105_2ghz4_3d_eabs_sphere.png}
		\caption{Simulation \SI{2.4}{\giga\hertz}}
	\end{subfigure}
	\begin{subfigure}[b]{0.48\textwidth}
		\includegraphics[width=1\textwidth]{../fig/plt/crazy_stuff_l4_pcb_v2c_laptop_1a_105_5ghz_3d_eabs_sphere.png}
		\caption{Simulation \SI{5.0}{\giga\hertz}}
	\end{subfigure}

	\begin{subfigure}[t]{0.49\textwidth}
	 	\includegraphics[width=1\textwidth]{../fig/plt/2_4GHz_justsphere.jpg}
		\caption{Messung \SI{2.4}{\giga\hertz}}
	\end{subfigure}
	\begin{subfigure}[t]{0.49\textwidth}
		\includegraphics[width=1\textwidth]{../fig/plt/5GHz_just_sphere.jpg}
		\caption{Messung \SI{5.0}{\giga\hertz}}
	\end{subfigure}
	\caption[Vergleich der 3D Abstrahlung aus Simulation und Messung]{
		Vergleich der 3D Abstrahlung aus Simulation und Messung.
		Die Abbildung zeigt die Abstrahlung für $\vec{E_{tot}}$
		bei gleicher Ansicht für die beiden Frequenzen
		\SI{2.4}{\giga\hertz} und \SI{5.0}{\giga\hertz}.}
\end{figure}

\clearpage
\subsection{Abstrahlung 1D}

\begin{figure}[h!]
	\centering
	\begin{subfigure}[b]{0.48\textwidth}
		\includegraphics[width=1\textwidth]{../fig/plt/crazy_stuff_l4_pcb_v2c_laptop_1a_105_2ghz4_eabs_phi90-trim.png}
		%\includegraphics[width=1\textwidth]{../fig/plt/crazy_stuff_l4_pcb_v2c_laptop_1a_105_eabs_phi90_2ghz4.png}
		\caption{Simulation \SI{2.4}{\giga\hertz}}
	\end{subfigure}
	\begin{subfigure}[b]{0.48\textwidth}
		\includegraphics[width=1\textwidth]{../fig/plt/crazy_stuff_l4_pcb_v2c_laptop_1a_105_5ghz0_eabs_phi90-trim.png}
		%\includegraphics[width=1\textwidth]{../fig/plt/crazy_stuff_l4_pcb_v2c_laptop_1a_105_eabs_phi90_5ghz.png}
		\caption{Simulation \SI{5.0}{\giga\hertz}}
	\end{subfigure}

	\begin{subfigure}[t]{0.49\textwidth}
		\includegraphics[width=1\textwidth]{../fig/plt/2G4_90phi_etot_dB.JPG}
	 	%\includegraphics[width=1\textwidth]{../fig/plt/2_4GHz_E_tot_curve.jpg}
		\caption{Messung \SI{2.4}{\giga\hertz}}
	\end{subfigure}
	\begin{subfigure}[t]{0.49\textwidth}
		\includegraphics[width=1\textwidth]{../fig/plt/5G0_90phi_etot_dB.JPG}
		%\includegraphics[width=1\textwidth]{../fig/plt/5GHz_E_tot_curve.jpg}
		\caption{Messung \SI{5.0}{\giga\hertz}}
	\end{subfigure}
\end{figure}


