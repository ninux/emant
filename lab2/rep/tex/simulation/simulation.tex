\section{Simulation}

\subsection{Erläuterung zu der Simulation}
Die Antenne wurde über mehrere Simulationsschritte verfeinert, so dass sie das gewünschte Verhalten aufzeigt. Gemäss Simulation sollte die Antenne einen Reflexionsfaktor von knapp -25dB aufweisen bei einer Frequenz von 2.4GHz. Bei 5 GHz sollte dieser bei -17dB liegen. Die reale Impedanz liegt gemäss Simulation bei 2.4 GHz bei 43 Ohm, der Imaginäranteil bei 10 Ohm kapazitiv. Bei 5 GHz liegt der Realanteil bei 40 Ohm und der Imaginäranteil bei 6 Ohm.


\clearpage
\subsubsection{Reflexionsfaktor $S_{11}$}
\begin{figure}[h!]
	\centering
	\includegraphics[width=0.7\textwidth]{../fig/plt/crazy_stuff_l4_pcb_v2c_laptop_1a_105_S11_2.png}
	\caption{Reflexionsfaktor $S_{11}$}
\end{figure}

%\clearpage
\subsubsection{Impedanz}
\begin{figure}[h!]
	\centering
	\includegraphics[width=0.7\textwidth]{../fig/plt/crazy_stuff_l4_pcb_v2c_laptop_1a_105_Widerstand_1.png}
	\caption{Impedanz}
\end{figure}

\subsubsection{Elektrisches Feld}
\begin{figure}[htbp]
	\begin{center}
		\begin{subfigure}[t]{0.49\textwidth}
			\begin{center}
				\includegraphics[width=0.94\textwidth]{../fig/plt/crazy_stuff_l4_pcb_v2c_laptop_1a_105_2ghz4_3d_electric_field_xy.png}
				\caption{bei 2.4 GHz}
				\label{fig:electricfield_2_4}
			\end{center}
		\end{subfigure}
		\begin{subfigure}[t]{0.49\textwidth}
			\begin{center}
				\includegraphics[width=1\textwidth]{../fig/plt/crazy_stuff_l4_pcb_v2c_laptop_1a_105_5ghz_3d_electric_field_xy.png}
				\caption{bei 5.0 GHz}
				\label{fig:electricfield_5_0}
			\end{center}
		\end{subfigure}
		\caption{Elektrisches Feld}
		\label{fig:electricfield}
	\end{center}
\end{figure}

\subsubsection{Oberflächenstromdichte}
\begin{figure}[htbp]
	\begin{center}
		\begin{subfigure}[t]{0.49\textwidth}
			\begin{center}
				\includegraphics[width=0.94\textwidth]{../fig/plt/crazy_stuff_l4_pcb_v2c_laptop_1a_105_2ghz4_3d_surface_current_density_xy.png}
				\caption{bei 2.4 GHz}
				\label{fig:currentdensity_2_4}
			\end{center}
		\end{subfigure}
		\begin{subfigure}[t]{0.49\textwidth}
			\begin{center}
				\includegraphics[width=1\textwidth]{../fig/plt/crazy_stuff_l4_pcb_v2c_laptop_1a_105_5ghz_3d_surface_current_density_xy.png}
				\caption{bei 5.0 GHz}
				\label{fig:currentdensity_5_0}
			\end{center}
		\end{subfigure}
		\caption{Oberflächenstromdichte}
		\label{fig:currentdensity}
	\end{center}
\end{figure}


\subsubsection{Fernfeld bei 2.4 GHz}
In die Simulation wurde ein Laptopmodell miteinbezogen, in welchem der USB-Dongle eingesteckt war. Der Bildschirm wurde mit einem Öffnungswinkel von 105° simuliert.

\begin{figure}[h!]
	\centering
	\begin{subfigure}[b]{0.96\textwidth}
		\includegraphics[width=1\textwidth]{../fig/plt/crazy_stuff_l4_pcb_v2c_laptop_1a_105_2ghz4_3d_farfield_eabs_xyz.png}
		\caption{$\vec{E}_{\mathrm{abs}}$}
	\end{subfigure}
	
	\begin{subfigure}[b]{0.48\textwidth}
		\includegraphics[width=1\textwidth]{../fig/plt/crazy_stuff_l4_pcb_v2c_laptop_1a_105_2ghz4_3d_farfield_ephi_xyz.png}
		\caption{$\vec{E}_{\varphi}$}
	\end{subfigure}
	\begin{subfigure}[b]{0.48\textwidth}
		\includegraphics[width=1\textwidth]{../fig/plt/crazy_stuff_l4_pcb_v2c_laptop_1a_105_2ghz4_3d_farfield_etheta_xyz.png}
		\caption{$\vec{E}_{\Theta}$}
	\end{subfigure}
	\caption{$\vec{E}$ Fernfeldanalyse (3D) bei 2.4 GHz}
\end{figure}

%\begin{figure}[htbp]
%	\begin{center}
%		\begin{subfigure}[t]{0.49\textwidth}
%			\begin{center}
%\includegraphics[width=1\textwidth]{../fig/plt/crazy_stuff_l4_pcb_v2c_laptop_1a_105_2ghz4_3d_farfield_eabs_xyz.png}
%				\caption{bei 2.4 GHz}
%				\label{fig:farfield_abs_2_4}
%			\end{center}
%		\end{subfigure}
%		\begin{subfigure}[t]{0.49\textwidth}
%			\begin{center}
%	\includegraphics[width=1\textwidth]{../fig/plt/crazy_stuff_l4_pcb_v2c_laptop_1a_105_5ghz_3d_farfield_eabs_xyz.png}
%				\caption{bei 5.0 GHz}
%				\label{fig:farfield_abs_5_0}
%			\end{center}
%		\end{subfigure}
%		\caption{$\vec{E}_{\mathrm{abs}}$ Fernfeldanalyse (3D)}
%		\label{fig:farfield_abs}
%	\end{center}
%\end{figure}


\clearpage
\subsubsection{Fernfeld bei 5 GHz}
\begin{figure}[h!]
	\centering
	\begin{subfigure}[b]{0.96\textwidth}
		\includegraphics[width=1\textwidth]{../fig/plt/crazy_stuff_l4_pcb_v2c_laptop_1a_105_5ghz_3d_farfield_eabs_xyz.png}
		\caption{$\vec{E}_{\mathrm{abs}}$}
	\end{subfigure}

	\begin{subfigure}[b]{0.48\textwidth}
		\includegraphics[width=1\textwidth]{../fig/plt/crazy_stuff_l4_pcb_v2c_laptop_1a_105_5ghz_3d_farfield_ephi_xyz.png}
		\caption{$\vec{E}_{\varphi}$}
	\end{subfigure}
	\begin{subfigure}[b]{0.48\textwidth}
		\includegraphics[width=1\textwidth]{../fig/plt/crazy_stuff_l4_pcb_v2c_laptop_1a_105_5ghz_3d_farfield_etheta_xyz.png}
		\caption{$\vec{E}_{\Theta}$}
	\end{subfigure}
	\caption{$\vec{E}$ Fernfeldanalyse (3D) bei 5.0 GHz}
\end{figure}


